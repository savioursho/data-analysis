% 参考にしたサイト
% https://mathlandscape.com/latex-amsthm/
% https://qiita.com/rityo_masu/items/5ef248024b294d72799a
% https://note.com/dragon_notepad/n/ne2a1392cf8ce
% https://lualatexlab.blog.fc2.com/blog-entry-43.html
% https://qiita.com/Hdan/items/8c59a7e0a3215ae32d74

\documentclass[]{ltjsarticle}

% 図
\usepackage{graphicx}

% 箇条書き
\usepackage{outlines}

% 数学系パッケージ
\usepackage{amsmath}
\usepackage{mathtools}
\usepackage{amsthm}
\usepackage{amsfonts}
\usepackage{physics}

% 圏点
\usepackage{pxrubrica}

% コメント
\usepackage{comment}

% 定理環境
\usepackage{tcolorbox}
\tcbuselibrary{breakable, skins, theorems}

% hyperref
% これは最後に置く
\usepackage{bookmark}
\usepackage{cleveref}
\hypersetup{
  unicode,
  bookmarksnumbered=true,
  hidelinks,
  colorlinks=true,
  linkcolor=blue,
  citecolor=red,
  }

% \usepackage{cleveref}より後に書く  
\newtcbtheorem[number within = section, crefname={定理}{定理}]{thm}{定理}{breakable = true}{thm}
\newtcbtheorem[number within = section, crefname={定義}{定義}]{dfn}{定義}{breakable = true, colframe=blue}{dfn}
\newtcbtheorem[crefname={分布}{分布}]{dist}{分布}{breakable = true, colframe=blue}{dist}
\newtcbtheorem[number within = section, crefname={例}{例}]{exm}{例}{breakable = true}{exm}


\crefname{equation}{式}{式}% {環境名}{単数形}{複数形} \crefで引くときの表示
\crefname{figure}{図}{図}% {環境名}{単数形}{複数形} \crefで引くときの表示
\crefname{table}{表}{表}% {環境名}{単数形}{複数形} \crefで引くときの表示

\crefname{section}{第}{第}
\creflabelformat{section}{#2#1章#3}
\crefname{subsection}{第}{第}
\creflabelformat{subsection}{#2#1節#3}

\numberwithin{equation}{section}